\section{Aufgabenstellung}
\label{sec:aufgabenstellung}

Bei der Anwendung kommen

Im folgenden wird zunächst zur Abgrenzung dieser Arbeit innerhalb des
Referenzsystems, der grundelegende Aufbau des Referenzsystems beschrieben. 

\subsection{Der Roboter}
\label{sub:roboter}

Auf der Ebene der Hardware steht der mobile Roboter. Dieser hat das Ziel sich
genau zu lokalisieren um Referenzwerte für die montierten, mobilen Sensorknoten zu
liefern. Da er sich während der Testfahrten indoor bewegt, kann er
für seine Lokalisierung keine satelliten-gestützten Lokalisierungssysteme wie
GPS verwenden, aus den in \ref{motivation} genannten Gründen. Zwei weitere
Ansätze, um die Bewegung des Roboters nachzuvollziehen und somit Positionsdaten
zu gewinnen sind:

\begin{itemize}
  \item Inertial Navigation
  \item Odometrie
\end{itemize}

Bei der intertial Navigation wird mithilfe von Beschleunigungs- und
Gyromessungen auf die ausgeführte Bewegung geschlossen. Diese Messungen lassen
sich druch \gls{MEMS}, die in einer 
\gls{IMU} zusammengefasst werden durchführen.
\gls{MEMS} werden in großen Stückzahlen produziert und sind kostengünstig.
Typischerweise sind aber die Messungen, auch bei Stillstand, durch Jitter belastet.
Dieser kann zwar durch geeignete Filter geglättet werden, lässt sich allerdings
nicht ganz ausschließen. Auf längere Strecken entsteht durch die Aufsummierung
der Fehler ein Drift, fort von der tatsächlichen Position.

Bei der Odometrie, werden die Antriebsdaten ausgewertet, um auf die Bewegung
des Roboters zu schließen. Geht man davon aus, dass der Untergrund, auf dem der
Roboter fährt, für dessen Räder geeignet ist und die Räder nicht wegen
beispielsweise mangelnder Bodenhaftung stark durchdrehen. So kann sie auf
kurzen Strecken sehr genaue Abschätzungen liefern. Allerdings ist auch eine
Odometriemessung stets mit einem Fehler behaftet. Dieser Fehler summiert sich
über die Zeit auf und die geschätzte Position weicht immer weiter von der
tatsächlichen ab. Man hat also auf längeren Strecken ebenfalls mit einem Drift
zu rechnen.

Beide Methoden haben gemeinsam, dass sie unabhängig von Informationen aus der
Umgebung des Roboters arbeiten. Somit können sie allerdings den beschriebenen
Drift in der Lokalisierung niemals korrigieren, da sie nicht die Positions auf
Plausibilität mit der Umgebung abgleichen.  Für das Referenzsystem ist Drift
aber nicht akzeptabel. Aus diesem Grund erfasst der Roboter Abstände zu
Hindernissen seiner Umgebung mithilfe einer \textit{Microsoft Kinect}. Die
Kinect erstellt mithilfe eines, im infrarot Bereich gestrahltem, optischen
Musters ein Tiefenbild.  Die Reichweite liegt dabei bei maximal 10 Meter
Entfernung bei einem Blickwinkel von ca $59^{\circ}$.
%TODO Quelle anbringen
Test haben gezeigt, dass die Genauigkeit der Tiefehmessung mit zunehmender
Entfernung abnimmt. Im Nahbereich von zwei Metern aber überraschend Präzise
Abstandsauflösung im Zentimeterbreich ermöglicht.  Außerdem verfügt der Roboter
über eine Karte der Testumgebung. Diese Karte in Kombination mit der
\textit{Microsoft Kinect} ermöglicht während einer Testfahrt eine Lokalisierung
durch folgende entscheidende Schritte durchzuführen:

\begin{enumerate}
  \item Abschätzung der derzeitigen Pose durch Odometrie
  \item Abgleich mit Karte und Korrektur der Pose
\end{enumerate}

Zum Abtasten der Umgebung hätte alternativ auch ein Laserscanner gewählt werden
können, welcher hohe Reichweite mit hoher Genauigkeit kombiniert. Dies wäre
aber entgegen den Ziele des Referenzsystems, mit zu hohen Anschaffungskosten
verbunden gewesen. Im Sinne günstiger Kosten wurde schlussendlich ein
sogenannter \textit{TurtleBot} gebaut. Dies ist ein von \textit{WillowGarage}
spezifizierter low-cost Roboter.
%TODO TurtleBot specs referenzieren?
Im Kern besteht dieser aus einem \textit{Roomba} Staubsaugerroboter von \textit{iRobot}, 
einer \textit{Microsoft Kinect} und einem Tragegerüst. Das Tragegerüst dient als
Abstellfläche für einen Laptop und bietet im Anwendungsfall des Referenzsystems
Platz zum Montieren der Sensorknoten.

%TODO Figure TurtleBot einfügen
%\begin{figure}[<+htpb+>]
%  \begin{center}
%    \psfig{figure=<+eps file+>}
%  \end{center}
%  \caption{<+caption text+>}
%  \label{fig:<+label+>}
%\end{figure}<++>

Die Aspekte der Software, zum Betrieb des Roboters, wurde innerhalb einer anderen
Bachelorarbeit, ebenfalls im Rahmen der Entwicklung des Referenzsystems, ausführlich 
erarbeitet.
%TODO Simon referenzieren?

Zusammenfassend kann man in dieser Hinsicht festellen, dass zum autonomen Fahren
Programme des \gls{ROS} genutzt werden sowie eine Software implementiert wurde,
die den Roboter vorgegebene Wegpunkte abfahren lässt und dabei gesammelte
Lokalisierungsdaten innerhalb von \gls{ROS} bereitstellt. Genauer wird im Teil
\ref{implementierung} auf die Funktionsweise von \gls{ROS} eingegangen.


