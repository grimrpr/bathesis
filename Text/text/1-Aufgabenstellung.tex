\section{Aufgabenstellung}
\label{sec:aufgabenstellung}

Im folgenden wird der grundelegende Aufbau des Referenzsystems beschrieben und
wie sich Pathcompare in dessen Topologie einfügt. 

Auf der Ebene der Hardware steht der mobile Roboter. Dieser hat das Ziel sich
genau zu lokalisieren um Referenzwerte für die montierten, mobilen Sensorknoten zu
liefern. Da er sich während der Testfahren natürlich indoor bewegt, kann er
für seine Lokalisierung keine satelliten-gestützten Lokalisierungssysteme wie
GPS verwenden, aus den in \ref{motivation} genannten Gründen. Zwei weitere
Ansätze, um die Bewegung des Roboters nachzuvollziehen und somit Positionsdaten
zu gewinnen sind:

\begin{itemize}
  \item Inertial Navigation
  \item Odometrie
\end{itemize}

Bei der intertial Navigation wird mithilfe von Beschleunigungs- und
Gyromessungen auf die ausgeführte Bewegung geschlossen. Diese Messungen lassen
sich druch \gls{MEMS}, die in einer 
\gls{IMU} zusammengefasst werden durchführen.
\gls{MEMS} werden in großen Stückzahlen produziert und sind kostengünstig.
Typischerweise sind aber die Messungen, auch bei Stillstand, durch Jitter belastet.
Dieser kann zwar durch geeignete Filter geglättet werden, lässt sich allerdings
nicht ganz ausschließen. Auf längere Strecken entsteht durch die Aufsummierung
der Fehler ein Drift, fort von der tatsächlichen Position.

Bei der Odometrie, werden die Antriebsdaten ausgewertet, um auf die Bewegung
des Roboters zu schließen. Geht man davon aus, dass der Untergrund, auf dem der
Roboter fährt, für dessen Räder geeignet ist und die Räder nicht wegen
beispielsweise mangelnder Bodenhaftung stark durchdrehen. So kann sie auf
kurzen Strecken sehr genaue Abschätzungen liefern. Allerdings ist auch eine Odometriemessung
stets mit einem Fehler behaftet. Dieser Fehler summiert sich über die Zeit auf
und die geschätzte Position weicht immer weiter von der tatsächlichen ab. Man
hat also auf längeren Strecken ebenfalls mit einem Drift zu rechnen.

Beide Methoden haben gemeinsam, dass sie unabhängig von Informationen aus der
Umgebung des Roboters arbeiten. Somit können sie allerdings den beschriebenen
Drift in der Lokalisierung niemals korrigieren, da sie nicht die Positions auf
Plausibilität mit der Umgebung abgleichen. 
Für das Referenzsystem ist Drift aber nicht akzeptabel. Aus diesem Grund
erfasst der Roboter Abstände zu Hindernissen seiner Umgebung mithilfe einer
\textit{Microsoft Kinect}. Die Kinect erstellt mithilfe eines, im infrarot
Bereich, gestrahltem optisches Muster ein Tiefenbild. 
Die Reichweite liegt bei maximal 10 Meter Entfernung mit
einem Blickwinkel von ca $59^{\circ}$. Test haben gezeigt, dass die Genauigkeit der
Tiefehmessung mit zunehmender Entfernung abnimmt, aber im Nahbereich
überraschend Präzise ist.
Außerdem verfügt der Roboter über eine Karte der
Testumgebung. Während einer Testfahrt werden für
eine Lokalisierung letzendlich folgende entscheidende Schritte ausgeführt:

\begin{itemize}
  \item Abschätzung der derzeitigen Pose durch Odometrie
  \item Abgleich mit Karte und Korrektur der Pose
\end{itemize}

Zum Abtasten der Umgebung hätte alternativ auch ein Laserscanner gewählt werden
können, dies wäre aber entgegen der Ziele des Referenzsystems mit zu hohen
Anschaffungskosten verbunden gewesen. Im Sinne günstiger Kosten wurde 
letzendlich ein sogenannter \textit{TurtleBot} gebaut. Dies ist ein
von \textit{WillowGarage} spezifizierter low-cost Roboter. Im Kern
besteht dieser aus einem \textit{Roomba} Staubsaugerroboter von \textit{iRobot}, einer
\textit{Microsoft Kinect} und ein Tragegerüst. Das Tragegerüst dient als
Abstellfläche für einen Laptop und bietet im Anwendungsfall des Referenzsystems
Platz zum Montieren der Sensorknoten.

%TODO Figure TurtleBot einfügen
%\begin{figure}[<+htpb+>]
%  \begin{center}
%    \psfig{figure=<+eps file+>}
%  \end{center}
%  \caption{<+caption text+>}
%  \label{fig:<+label+>}
%\end{figure}<++>

Softwareseitig wird der Roboter innerhalb von \gls{ROS} betrieben. Dieser
Aspekt wurde innerhalb einer anderen Bachelorarbeit im Rahmen der Entwicklung des
Referenzsystems ausführlich erarbeitet. 
sich mit der Einrichtung der Software für
den Roboter sowie mit dem 
