\chapter*{\abstractname}
% delete german abstract when writing in english
\section*{Zusammenfassung} Diese Arbeit steht im Kontext zum Aufbaus eines
Referenzsystems, welches in der Lage ist sich mobil indoor zu bewegen und dabei
möglichst genau zu lokalisieren.  Der Zweck dieses Referenzsystems ist es,
Lokalisierungen als Referenz zur Verfügung zu stellen, um die Genauigkeit von
Lokalisierungen mobiler Sensorknoten, die in einem 
\gls{WSN} organisiert sind, zu ermitteln. 

Diese Arbeit steht in Kontext zum Aufbau eines Referenzsystems zur indoor
Lokalisierung. Dieses Referenzsystem soll es ermöglichen die Genauigkeit von
indoor Lokalisierungen, durchgeführt mit mobilen Sensorknoten eines \gls{WSN},
zu ermitteln. Die Grundlage des Referenzsystems ist dabei ein mobiler Roboter,
welcher autonom, vorgegebene Wegpunkte in einer Karte abfährt. Bei einer solchen
Fahrt zeichnet der Roboter sowie an ihm befestigte Sensorknoten einen Pfad
durch regelmäßige Lokalisierung auf. Aufbauend darauf, geht es im Rahmen dieser
Arbeit um die Implementierung eines Analysewerkzeugs namens
\textit{Pathcompare}, welches ermöglicht, die dabei entstandenen Pfaddaten
zusammenzuführen, für den Tester aufzuwerten und zu visualisieren. Neben
mittleren Abstand zum gewählten Referenzpfad werden Parameter wie
Pfadlänge, Anzahl der Pfadpunkte, empirische Varianz und eine Liste der größten
Abweichungen angezeigt. Alle Daten können als \gls{CSV} exportiert werden.
\textit{Pathcompare} ist in das \gls{ROS} integriert und so entwickelt, dass es
über Plug-ins erweitert und angepasst werden kann. 


% always write an english abstract in addition to a german one
\section*{Abstract} This thesis is associated with the development of a indoor
localization reference system. The aim of this reference system is to evaluate
the precision of mobile sensor node localization data. The sensor nodes are
organized in a wireless sensor network. The basis of the reference system is a
mobile robot, that is able to autonmously navigate to given waypoints in a map.
While moving, the robot and sensor nodes mounted on it, generate path data by
continously localizing. This work is about the creation of an analysing tool
named \textit{Pathcompare} that is used to merge the path data of different
sources and visualize them for a tester. The software shows the median distance
of a path to the given reference path, the overall pathlength, total number of
points per path, variance and also a list of the greates distances to the
reference path.  All results can be exported in a \gls{CSV} file.
\textit{Pathcompare} is integrated into the \gls{ROS} and can be extended via a
Plug-in mechanism.
