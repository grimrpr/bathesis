\chapter*{\abstractname}
% delete german abstract when writing in english
\section*{Zusammenfassung} Diese Arbeit steht im Kontext zum Aufbaus eines
Referenzsystems, welches in der Lage ist sich mobil indoor zu bewegen und dabei
möglichst genau zu lokalisieren.  Der Zweck dieses Referenzsystems ist es,
Lokalisierungen als Referenz zur Verfügung zu stellen, um die Genauigkeit von
Lokalisierungen mobiler Sensorknoten, die in einem \textit{Wireless Sensor
Networks} \gls{WSN} organisiert sind, zu ermitteln. Das Referenzsystem ist
durch einen Roboter realisiert, welcher autonom vorgegebene Wegpunkte in einer
Karte abfährt. Bei einer solchen Fahrt zeichnet der Roboter sowie an ihm
befestigte Sensorknoten einen Pfad durch regelmäßige Lokalisierung auf. In
dieser Arbeit geht es um die Implementierung eines Analysewerkzeugs namens
\textit{Pathcompare}, welches ermöglicht, die Pfaddaten zusammenzuführen, für
den Tester aufzuwerten und zu visualisieren. Neben mittleren Abstand (Median)
zum gewählten Referenzpfad werden Parameter wie Pfadlänge, Anzahl der
Pfadpunkte und eine Liste der größten Abweichungen angezeigt.
\textit{Pathcompare} ist dabei in das \gls{ROS} integriert und so entwickelt
dass es über Plug-ins erweitert und angepasst werden kann. 


% always write an english abstract in addition to a german one
\section*{Abstract} This thesis is associated with the development of a
referece system that has the ability to localize itself. The reason for the
development of such a system is to provide localization data. This data is then
used as reference data for localization data of mobile sensor nodes organized
in a \textit{Wireless Sensor Network} \gls{WSN} inorder to evaluate the
precision of their localization measurements. Such a reference system was build
in the form of a mobile robot that is able to autonomously navigate to given
waypoints.  While moving, the robot and sensor nodes mounted on it, generate
path data by continously localizing. This work is about the creation of an
analysing tool named \textit{Pathcompare} that is used to merge the path data
of different sources and visualize them for a tester. The software shows the
median distance of a path to the given reference path, the overall pathlength,
total number of points per path and als a list of the greates distances to the
reference path. \textit{Pathcompare} is integrated into the \gls{ROS} and can
be extended via a Plug-in mechanism.
