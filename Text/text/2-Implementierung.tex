\chapter{Pathcompare - Implementierung}
\label{sec:implementierung}
\section{Technischer Rahmen}

\subsection{Robot Operating System - ROS}
\begin{itemize}
  \item Warum brauchen wir ROS?
  \item Wer entwickelt ROS? (Stanford University \& WillowGarage \& community)
  \item Was sind die Grundeigenschaften 
  \item multitool Ansatz
  \item strukturierte Kommunikation zwischen den Tools
  \item free and open source
  \item multilingual 
  \item Begrifflichkeiten des ROS (Stack, Package, Node, Master, Topic, Message)
  \item Topologie (figure: Roboter, Master, Pathcompare , Sensorknoten)
  \item standard Buildsystem cmake
\end{itemize}

\subsection{Qt}
\begin{itemize}
  \item Warum brauchen wir Qt
  \item GUI framework, native in C++ (auch Implementierungen für Java, Ruby,..)
  \item bietet auch zahlreiche nicht GUI spezifische Funktionalität
  \item cross platform
  \item native UI rendering
  \item UI designing with Qt Designer
  \item bereichert C++ durch embedded Macros (z.B. signal\&slot) -> moc compiler
  \item standard Buildsystem native qmake 
  \item Unterstützung für cmake gegeben, für große Projekte sogar empfohlen
\end{itemize}


\section{Design der Software}
Der Hauptfokus beim Design der GUI lag darauf, einfache Benutzung und
übersichtliche Testdatendarstellung, für den Nutzer zu gewährleisten.
Außerdem sollte die Software leicht erweiterbar sein.

\subsection{Überblick Gesamtsystem}
%screenshots: TopicView  sowie Gesamtansicht mit
\begin{itemize}
  \item Rahmen mit Topic TreeView
  \item Anbindung an ROS durch ROSManager
  \item TabFlächen für einzelne PlugIns
  \item PluginLoader lädt Plugins
\end{itemize}

\subsection{Plug-in Main Compare}
%screenshot des Plugins
\begin{itemize}
  \item Funktionen erklären
  \item Referenz Pfad Selektion
  \item Export (+Format der csv Datei)
  \item Tabellenansicht erklären
\end{itemize}

\subsubsection{Pfadvergleichsverfahren}

%bild test mit Tickgeber

%bild variante ohne tick

%bild nav_msgs/Path message
\subsection{Plug-in Konzept allgemein}
\begin{itemize}
  \item allgemein Qt Plugins
  \item Vorstellen der Interfaces zum Schreiben eigener Plug-ins für Pathcompare
  \item Beispiel Camera Plugin
\end{itemize}
