\chapter{Pathcompare - Implementierung}
\label{sec:implementierung}
\section{Technischer Rahmen}

\subsection{Robot Operating System - ROS}

Innerhalb des Referenzsystems wird \gls{ROS} im Zusammenhang mit der
Steuerung des Roboters genutzt und um die, während einer Testfahrt gewonnenen Pfaddaten,
zur Verfügung zu stellen. Im Folgenden wird genauer auf die Fähigkeiten und
Ziele von \gls{ROS} eingegangen.

Obwohl der Name zunächst anderes vermuten lässt ist \gls{ROS} kein Betriebssystem im
klassischem Sinne. Es ist ein Framework, welches auf ein Betriebssystem
angewiesen ist um ausgeführt werden zu können. Es bietet aber Funktionalitäten,
die abstrahiert betrachtet Betriebsystemfunktionen ähneln. Charakteristisch ist
hierbei \gls{ROS} Fähigkeit lokal- oder nichtlokalausgeführte Programme zu Verbinden und
eine strukturierte Kommunikation zwischen diesen zu ermöglichen. Die
wesentlichen Elementareigenschaften der Grundphilosophie sind:

\begin{itemize}
  \item multi-tool Ansatz
  \item peer-to-peer Kommunikation
  \item keine feste Bindung an Programmiersprache
  \item frei und Open-Source
\end{itemize}

Multi-tool Ansatz bedeutet, dass \gls{ROS} die Fähigkeiten verschiedener
Programme und Libraries zur Verfügung stellt. Diese sind jedoch nicht fest in
den Kern von \gls{ROS} eingebaut sondern modular integriert. Als analoges Beispiel in
Hinblick auf Betriebssysteme kann man \gls{ROS} also als einen Mikrokernel
verstehen. Dies bietet den Vorteil, dass \gls{ROS} selbst vergleichsweise klein
ist und nur wirklich gebrauchte tools geladen werden müssen.
Die peer-to-peer Kommunikation bezieht sich auf die Kommunikation
zwischen diesen, in \gls{ROS} integrierten, Modulen. Diese wird durch den \gls{ROS}
Kern gesteuert. Der Kern von \gls{ROS} ist nativ in C++ implementiert, es
existieren jedoch bereits Portierungen in andere Sprachen wie Python, Octave
und Lisp um die \gls{ROS}-\gls{API} einer größeren Zahl von Entwicklern und
Projekten die Nutzung zu ermöglichen. Weitere Portierungen sollen sich in der Implementierung befinden.

%TODO hier Referenz zu ROS paper bringen oder später?

\gls{ROS} ist darüberhinaus frei Verfügbar und Open-Source. Man kann beliebige
Programme, als Module zur Erweiterung und Nutzung von \gls{ROS} hinzufügen, wie
es auch im Rahmen des Referenzsystems geschehen ist. Bei allgemeinem Nutzem und gegebener
Pflege, besteht die Möglichkeit, dass diese offiziell zu ROS hinzugefügt werden.

Um die konkreten Abläufe und Komponenten innerhalb von \gls{ROS} 
veranschaulichen zu können und damit auch den Bezug zu Pathcompare herstellen
zu können, ist es zunächst erforderlich die Begrifflichkeiten innerhalb von
\gls{ROS} zu klären. Im Folgendem werden diese aufgezeigt.

%TODO figure ROS Komponenten hinzufügen und im Text referenzieren
Im Zentrum von ROS steht der sogenannte \textit{master}. Dieser wird als
einzelne Instanz gestartet und wartet dann darauf, dass sich tools, die im
Kontext von \gls{ROS} gestartet werden, bei ihm anmelden. Ein
gestartetes tool wird innerhalb von \gls{ROS} als \textit{node} bezeichnet. 
Ist der \textit{master} nicht gestartet können auch keine \textit{nodes}
gestartet werden. Die \textit{nodes} sind also alle zunächst auf Kommunikation mit dem \textit{master}
angewiesen. Diese Kommunikation kann lokal oder nichtlokal ausgeführt werden,
d.h. der \textit{master} kann sich auch auf einem anderen Rechner wie der
\textit{node} befinden solange eine http Verbindung zwischen beiden hergestellt
werden kann. Denn das Anmelden des \textit{nodes} beim master erfolgt über einen
\gls{XML-RPC}, getragen vom http. Für den tool Entwickler auf Anwendungsebene ist diese
Kommunikation zur Anmeldung allerding völlig unsichtbar und er muss sich nicht darum kümmern.
Das Ausführen von \textit{nodes} auf unterschiedlichen Rechnern ist natürlich ebenso
möglich und wie zuvor bereits erwähnt eines der Kernfunktionen von \gls{ROS}.
Dies lässt sich vorteilhaft ausnutzen durch zum Beispiel:
\begin{itemize}
  \item Verteilung oder Auslagerung rechenintensiver \textit{nodes} auf potente Hardware
  \item Zusammenführenung von an unterschiedlichen Stellen gewonnener Daten.
\end{itemize}

So muss beispielsweise ein mobiler Roboter Bilderkennungs Aufgaben nicht selbst
ausführen, sondern kann diese an einen \textit{node} weiterleiten der auf einem
Rechencluster ausgeführt wird. Die zweite genannte Möglichkeit ist auch besonders im Bezug auf
diese Arbeit wichtig, da Pfaddaten des Roboters und der zu testenden
Sensorknoten für \textit{Pathcompare} verfügbar gemacht werden müssen.
Die Kommunikation zwischen Nodes erfolgt über sogenannte
\textit{messages}, diese enthalten die serialisierte Form der zu übertragenden Daten.
\gls{ROS} bietet in seinen Kernpakten bereits zahlreiche Definitionen für
unterschiedliche \textit{message} Typen, aber es ist auch
möglich eigene zu generieren und dies wird von
zahlreichen Paketen getan um Daten maßgeschneidert übertragen zu können. Einmal definierte \textit{message} Typen können
wiederum rekusiv in anderen \textit{message} Typen verwendet werden.
Ein Beispiel für eine message ist in \autoref{lst:transform} dargestellt; eine Transformation die in ROS geometry\_msgs
definiert ist. Sie besteht wie erkennbar aus den zwei Typen \textit{Vector3}
und \textit{Quaternion}. Letzterer beschreibt die Rotation und der erste die
Translation. Zusammengefügt ergibt dies eine Transformationsnachricht.

\begin{lstlisting}[caption=ROS transformation message, label=lst:transform]
geometry_msgs/Vector3 translation
  float64 x
  float64 y
  float64 z
geometry_msgs/Quaternion rotation
  float64 x
  float64 y
  float64 z
  float64 w
\end{lstlisting}

Soll ein \textit{node} seine \textit{messages} anderen \textit{nodes} 
senden können, so muss dies zunächst durch festlegen einer sogenannten \textit{topic} beim
\textit{master} angemeldet werden. Dann wird diese \textit{topic} für andere
\textit{nodes} im \gls{ROS} über den \textit{master} sichtbar.
Eine \textit{topic} besteht im wesentlichen aus zwei Teilen, einer
\textit{topic-id} und einem \textit{message-type}. Wobei der
\textit{message-type} angibt welcher Typ von \textit{messages} über diese
\textit{topic} verschickt wird. Die \textit{topic-id} dient zur eindeutigen
Identifikation innerhalb des \gls{ROS}. \textit{Nodes} welche \textit{messages}
einer \textit{topic} empfangen sollen, müssen diese \textit{topic} dann beim
\textit{master} abbonieren. In programmatischer Hinsicht wird beim Empfang
neuer Nachrichten innerhalb des \textit{nodes} eine festgelegte callback
Methode aufgerufen um die \textit{messages} zu bearbeiten. Treffen dabei
Nachrichten mit einer höheren Frequenz ein, als
abgearbeitet werden können, so kommt es irgendwann zu Verlusten wenn die
einstellbare Größe der \textit{message queue} überschritten wird.

Zwei weitere wichtige Begriffe in \gls{ROS} betreffen die Organisierung der
Dateien die zu den einzelnen tools gehören. Dies sind:

\begin{itemize}
  \item package
  \item stack
\end{itemize}

Ein \textit{package} beinhaltet den Code, Libraries sowie die ausführbare Datei eines
tools bzw. \textit{nodes}.
In \gls{ROS} sind für packages bestimmte Ordnerstrukturen und Dateien
festgelegt sodass mithilfe der von \gls{ROS} mitgebrachten tools \textit{packages}
leicht gebaut, gesucht und gestartet werden können. Beispielsweise basiert das 
\gls{ROS} build System auf cmake und so ist eine vorkonfigurierte
\textit{CMakeLists.txt} in jedem \textit{package} grundsätzlich enthalten. Eine Zusammenfassung mehrerer
\textit{packages} wird als \textit{stack} bezeichnet. 

\textit{Pathcompare} ist also im Bezug auf \gls{ROS}, während der Ausführung, ein
einzelner \textit{node} welcher \textit{topics} abboniert um \textit{messages}
zu empfangen. Es wird in einem
späteren Teil darauf eingegangen, welche \textit{messages} das genau sind.
Alle zum Kompilieren und Ausführen nötige Dateien sind dabei in zwei \textit{packages} namens \textit{pathcompare}
und \textit{pathcompareplugins} aufgeteilt.

%TODO figure einfügen referenzsystem im ros

\begin{itemize}
  \item Warum brauchen wir ROS?
  \item Wer entwickelt ROS? (Stanford University \& WillowGarage \& community)
  \item Was sind die Grundeigenschaften 
  \item multitool Ansatz
  \item strukturierte Kommunikation zwischen den Tools
  \item free and open source
  \item multilingual 
  \item Begrifflichkeiten des ROS (Stack, Package, Node, Master, Topic, Message)
  \item Topologie (figure: Roboter, Master, Pathcompare , Sensorknoten)
  \item plattformen: Ubuntu
  \item standard Buildsystem cmake
\end{itemize}

\subsection{Qt}

\textit{Pathcompare} ist darauf ausgelegt alle Informationen für den Nutzer in
einer Benutzeroberfläche (fortan als GUI bezeichnet) zu visualisieren.
Da die Anbindung an \gls{ROS} über C++ erfolgt, lag nahe auch die GUI in C++
umzusetzen. Dazu wurde das Qt Framework gewählt. Qt ist in C++ implementiert, wobei
allerdings auch Anbindungen für zahlreiche andere Sprachen wie z.B.
Java, C\#, Ruby oder Python existieren. Die Entwicklung von Qt begann 1991 und
%TODO Quelle anbringen
%http://my.safaribooksonline.com/0131872494/pref04?portal=oreilly#X2ludGVybmFsX0ZsYXNoUmVhZGVyP3htbGlkPTAxMzE4NzI0OTQveGlpaQ==
ist zum Zeitpunkt des Schreibens in der Version 4.7.4 verfügbar. Das Framework besteht dabei
mittlerweile nicht mehr nur aus reinen GUI Bibliotheken, sondern stellt auch
z.B. Netzwerk-, SQL- und andere Anwendungs-Bibliothken zur Verfügung.

Für die Entwicklung von \textit{Pathcompare} waren neben den GUI Bibliotheken,
besonders folgende Konzepte und Funktionen beim Entwickeln vorteilhaft: 

\begin{itemize}
  \item Signal-Slot Konzept
  \item Plattformunabhängigkeit
  \item Containerklassen mit praktischen Hilfsmethoden
  \item graphischer GUI Designer 
\end{itemize}

Auf einige dieser Punkte und deren Bezug zu Pathcompare wird nun kurz eingangen.

Das \textit{Signal-Slot} Konzept dient dazu Objekten, bestimmte Veränderungen an Objekten
mitzuteilen, welche über diese Änderung informiert werden sollen.
Es realisiert das Entwurfsmuster des \textit{Observer patterns}. 
\textit{Signal-Slot} erspart es dem Programmierer einen Verweis auf das beobachtende
Objekt, durch einen registrierenden Methodenaufruf beim aktualisierenden
Objekt, zu hinterlegen. Stattdessen emittiert, im Falle einer mitzuteilenden
Aktualisierung, das aktualisierende Object einen Methode, welche durch
einen Qt Makro als \textit{signal} ausgezeichnet ist. Bei Beobachter Objekten,
die auf dieses Signal reagieren sollen, wird dann eine als \textit{slot}
deklarierte Methode selbständig aufgerufen, sofern die Objekte, durch einen
Aufruf einer von Qt bereitgestellten, statischen Methode, verbunden wurden. 
Dieses Konzept vereinfacht die Entwicklung wegen seiner Flexibilität deutlich, allerdings ist der durch
Qt erzeugte Code, welcher bei Auflösung der Makros entsteht, aufblähend und
dadurch theoretisch langsamer bei der Ausführung als eine klassische callback Implementierung.
In der Praxis ist diese Verlangsamung aber unmerklich und wiegt nicht die
Vorteile für den Entwickler auf. Deshalb wurde das Konzept auch bei der Entwicklung von
\textit{Pathcompare} eingesetzt. 

Im Zusammenhang mit dem Signal-Slot Konzepts wurde bereits erwähnt, dass Qt C++
mit verschiedenen Makros erweitert. Diese Makros werden dabei nicht immer
direkt in gültigen C++ Code übersetzt, sondern dienen als Annotationen. Dies
hat Folgen für den Build Vorgang, denn die mit Annotationen versehenen Klassen
müssen dann zuerst, mit dem von Qt bereitgestelltem \gls{MOC} übersetzt werden.
Standardmäßig wird in \textit{Qt} das Build Programm qmake verwendet, welches
die nötigen Aufrufe des \gls{MOC} automatisch veranlasst. In Hinblick auf
\textit{Pathcompare} war aber eine Integration in das von \gls{ROS}
unterstützte Build verfahren cmake notwendig. Hierbei muss beachtet werden,
dass cmake die Dateien welche einen \gls{MOC} Aufruf notwendig machen, in der
derzeitigen Version, nicht selbständig erkennt. Diese müssen manuell in der
\textit{CmakeLists.txt} deklariert werden. Die sonstige Einbindung
\textit{Qt} spezifischer Libraries und deren Verlinkung mit der Applikation ist
in cmake leicht möglich. Somit war die Verwenung von Qt in
\textit{pathcompare} kein Hindernis für die Verwendung von cmake.
Die Nutzung von cmake wird darüberhinaus, für große Projekte, von Autoren im
Rahmen der Qt Dokumentation selbst empfohlen.
%TODO Quelle anbringen?

Neben dem \gls{MOC} existiert noch ein sogenannter \gls{UIC}. Dieser tritt im
Zusammenhang mit dem \textit{QtDesigner} in Erscheinung. Der
\textit{QtDesigner} ermöglicht es graphisch Qt GUIs zu erstellen. Dabei wird
eine XML datei vom Designer erstellt, welche den Aufbau der GUI abbildet.
Der UIC ist dann dafür verantwortlich diese Datei in C++ Klassen zu übersetzen.
Auch hier muss cmake veranlasst werden zunächst den gls{UIC} aufzurufen.

\begin{itemize}
  \item Warum braucht Pathcompare Qt?
  \item GUI framework, native in C++ (auch Implementierungen für Java, Ruby,..)
  \item bietet auch zahlreiche nicht GUI spezifische Funktionalität
  \item cross platform
  \item native UI rendering
  \item UI designing with Qt Designer
  \item bereichert C++ durch embedded Macros (z.B. signal\&slot) -> moc compiler
  \item standard Buildsystem native qmake 
  \item Unterstützung für cmake gegeben, für große Projekte sogar empfohlen
\end{itemize}


\section{Design der Software}
Der Hauptfokus beim Design der GUI lag darauf, einfache Benutzung und
übersichtliche Testdatendarstellung, für den Nutzer zu gewährleisten.
Außerdem sollte die Software leicht erweiterbar sein.

\subsection{Überblick Gesamtsystem}
%screenshots: TopicView  sowie Gesamtansicht mit
\begin{itemize}
  \item Rahmen mit Topic TreeView
  \item Anbindung an ROS durch ROSManager
  \item TabFlächen für einzelne PlugIns
  \item PluginLoader lädt Plugins
\end{itemize}

\subsection{Plug-in Main Compare}
%screenshot des Plugins
\begin{itemize}
  \item Funktionen erklären
  \item Referenz Pfad Selektion
  \item Export (+Format der csv Datei)
  \item Tabellenansicht erklären
\end{itemize}

\subsubsection{Pfadvergleichsverfahren}

%bild test mit Tickgeber

%bild variante ohne tick

%bild nav_msgs/Path message
\subsection{Plug-in Konzept allgemein}
\begin{itemize}
  \item allgemein Qt Plugins
  \item Vorstellen der Interfaces zum Schreiben eigener Plug-ins für Pathcompare
  \item Beispiel Camera Plugin
\end{itemize}
