\chapter{Fazit}
\label{sec:conclusion}
Pathcompare dient dazu, die Genauigkeit von Lokalisierungen mobiler
Sensorknoten in Bezug auf Lokalisierungen eines Referenzsystems abzubilden.
Die Grundlage für das Referenzsystem stellt ein mobiler Roboter dar, welcher
mithilfe des ROS arbeitet. Der Nutzer kann während einer
Testdurchführung relevante Informationen an in einer GUI übersichtlich beobachten. 
Zusätzlich wird durch die Plug-in Erweiterbarkeit von Pathcompare stets
ermöglicht, auf neue Testanforderungen zu reagieren oder Informationen
andersartig in der GUI zu visualisieren. Gleichzeitig werden beim Plug-in
Entwickler und dem Nutzer keinerlei tiefergehende ROS Kenntnisse vorausgesetzt,
da die Anbindung an ROS vollständig durch Pathcompare gekapselt wurde. 
Pathcompare hat seine geplante Grundfunktionalität in der derzeitigen Version
realisiert, aber es gibt dennoch denkbare Erweiterungen und Anpassungen welche die
Benutzbarkeit verbessern können. So könnte ein neues Plug-in beispielsweise
empfangene Pfaddaten zeichnen. Insgesamt ist eine solide Grundlage erstellt
worden, welche die Tests mit den Sensorknoten und dem Referenzsystem,
vereinfachen kann.

