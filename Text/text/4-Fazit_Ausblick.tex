\chapter{Fazit}
\label{sec:conclusion}
Pathcompare und Main Compare stellen in der derzeitigen Form ein Werkzeug dar, welches die
Durchführung der mit dem Referenzsystem geplanten Tests erleichtern kann. Dies wird vorallem
dadurch erreicht, dass es dem Nutzer ermöglicht wird, während der Testdurchführung, relevante
Informationen an einem Ort zu beobachten. Zusätzlich wird durch die Plug-in
Erweiterbarkeit stets die Möglichkeit offengehalten auf neue Testanforderungen
zu reagieren oder Informationen andersartig zu visualisieren. Gleichzeitig
werden beim Plug-in Entwickler und dem Nutzer keinerlei tiefergehende ROS
Kenntnisse vorausgesetzt, da die Anbindung an ROS vollständig durch Pathcompare
gekapselt wurde. Auch wenn Pathcompare seine geplante Grundfunktionalität in der
derzeitigen Version realisiert, gibt es dennoch denkbare Erweiterungen und
Anpassungen welche die Benutzbarkeit verbessern können. 
Außerdem sollten wie bereits im Abschnitt Anwendung angesprochen verstärkt
Defekt- und Benutzertests durchgeführt werden um die Benutzbarkeit abzusichern
und zu verbessern. Insgesamt wird aber eine solide Grundlage geliefert,
welche die Test, mit den Sensorknoten und dem Referenzsystem, vereinfachen kann.

