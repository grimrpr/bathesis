\chapter{Einleitung}
\label{cha:einleitung}
%==============================================================================

\section{Motivation}
\label{sec:motivation}
%==============================================================================

Systeme zur Lokalisierung werden für zahlreiche Zwecke genutzt und deren
Bedeutung wächst parallel zur Verbreitung immer neuer sogenannter \gls{LBS}.
Dies sind Anwendungen, die dem Nutzer auf Grundlage von Positionsdaten
Informationen generieren können. Für Anwendungen im Freien haben sich
Satelliten gestützte Lokalisierungssysteme, welche theoretisch hohe Genauigkeit
bieten können, etabliert. Als bekanntes Beispiel sei hier das
\textit{NAVSTAR-GPS} genannt, welches zivil und militärisch genutzt wird.
Allerdings ergeben sich auch Anwendungsumgebungen, in denen derartige Systeme
gar nicht, bzw. nur ungenau funktionieren oder bewusst z.B. aus Kostengründen
gemieden werden. Dies sind typischerweise Umgebungen in denen die
Satellitensignale zu stark gedämpft werden oder vor allem durch Reflexionen
bedingte Laufzeitverschiebungen, sich negativ auf die Genauigkeit auswirken,
wie z.B.: 

\begin{itemize} 
  \item innerhalb von Gebäuden (``indoor'') 
  \item im Untergrund (Tunnel, Höhlen u.ä.) 
  \item im Bereich dicht bebauter urbaner Gebiete
\end{itemize}

Um in solchen Umgebungen dennoch Lokalisierung zu ermöglichen, wurden und werden
viele theoretische Konzepte und konkrete Systeme entwickelt. In der
Arbeitsgruppe \gls{CST} an der \textit{FU-Berlin}, wurde dem Problem der
indoor Lokalisierung mit der Entwicklung eines \gls{WSN} basierten Systems
begegnet. Die dabei verwendete Infrastruktur und Lokalisierungsmethoden zielen
darauf ab, die Umgebung in welcher sich lokalisiert wird, nicht vorher mit
statischer Infrastruktur bestücken zu müssen. Das macht das System für spontane
oder kostenkritische Lokalisierungsszenarien attraktriv wie beispielsweise für:

\begin{itemize}
  \item Rettungseinsätze 
  \item Forschungsmissionen
  \item militärische Einsätze
\end{itemize}

So entstand vorausgehende Forschung der Arbeitsgruppe in diesem Bereich u.a. in
Kooperation mit der Berliner Feuerwehr. Die Sensorknoten des WSN sind in mobile
Knoten und Ankerknoten aufgeteilt. Das Ziel ist es mit einem mobilen Knoten
eine Lokalisierung durchzuführen. Die Grundlage dafür bieten die Ankerknoten,
dessen Position bekannt ist. Die mobilen Knoten ermitteln dann direkt mittels
der Ankerknoten oder inderekt über andere mobile Knoten ihre Position. Die
Lokalisierung kann dabei mithilfe verschiedener Algorithmen erfolgen. Da diese
Algorithmen untereinander und in Abhängigkeit verschiedener
Umgebungseigenschaften mit unterschiedlicher Präzision lokalisieren, ist es
notwendig Vergleiche zwischen ihnen anzustellen. Ein Weg dies zu tun sind
Simulationen der verschiedenen Algorithmen. Simulationen können kostengünstig
und schnelle Resultate liefern. Der Nachteil ist allerdings, dass sie stets
von realen Bedingungen abstrahieren. So kann es sein, dass eine durch
Abstraktion ausgeblendete Eigenschaft einer realen Umgebung, direkt oder
indirekt, die durch Simulation erwarteten Resultate verzerrt. Daher entstand
der Bedarf auch praktische Tests in einer indoor Umgebung durchzuführen und es
wurde die Aufgabe gestellt, ein Referenzsystem aufzubauen. Ein solches
Referenzsystem soll in einer indoor Umgebung präzise Lokalisierungen
durchführen, welche als Referenzwerte für Lokalisierungen der mobilen
Sensorknoten dienen. Das Referenzsystem ist mithilfe eines mobilen Roboters
realisiert worden, welcher durch Abfahren einer Strecke und stets wiederholende
Lokalisierung den gefahrenen Pfad aufzeichnet.  An ihm angebrachte mobile
Sensorknoten führen ebenfalls wiederholt Lokalisierungen durch. Im Kontext des
Referenzsystem wurde eine Software, gennannt Pathcompare, entwickelt.  Diese
ermöglicht es, Pfade des Referenzsystems und die der Sensorknoten, während
eines Tests, zusammenzuführen, um einem Tester eine Übersicht über die
erreichte Präzision der Lokalisierungen von Sensorknoten gegenüber dem
Referenzpfad aufzuzeigen. Pathcompare bietet dazu eine übersichtliche GUI und
kann durch Plug-ins in seiner Funktionalität erweitert werden.

\section{Struktur der Arbeit}
Die Arbeit ist in mehrere Abschnitte gegliedert, die unterschiedliche Aspekte
von Pathcompare erläutern. Zunächst wird im unmittelbar folgenden Abschnitt
``Aufgabenstellung'' die Zielsetzung von Pathcompare und der gewählte
Lösungsansatz aufgezeigt. Dazu wird auch auf die Gesamtstruktur des
Referenzsystems eingegangen und der mobile Roboter genauer beleuchtet.  Danach
folgt der Abschnitt ``Implementierung''. Hier wird in Detail auf Pathcompares
Bestandteile eingegangen. Dabei werden deren Funktionsweisen und
Designentscheidungen betrachtet. Ein wesentliches Element dieses Abschnitts ist
außerdem die Betrachtung von Pathcompares Plug-in Konzept und den
Voraussetzungen zum Schreiben eigener Pathcompare Plug-ins. Im Abschnitt
``Anwendung'' werden durchgeführte Tests von einzelnen Komponenten Pathcompares
betrachtet und nötige Voraussetzungen zum erfolgreichen Ausführen der Anwendung
aufgezeigt. Das ``Fazit'' fasst die wesentlichen Eigenschaften von Pathcompare
zusammen und verweist auf mögliche Verbesserungen und noch offene
Fragestellungen.

