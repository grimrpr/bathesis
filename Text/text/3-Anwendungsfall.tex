\chapter{Anwendung}
\label{sec:andwendung}
In diesem Abschnitt soll ein typischer Anwendugsfall für Pathcompare und dem
Main Compare Plugin beschrieben werden. 

%\section{Installation}
%
%\begin{itemize}
%  \item Betriebsysteme
%  \item ROS version - diamondback
%  \item package structure
%  \item build calls
%\end{itemize}

\section{Ausführung Pathcompare}

Pathcompare ist in die ROS packages pathcompare und pathcompareplugins
aufgeteilt. Die ausführbare Datei der Pathcompare Hauptanwendung befindet sich
im package pathcompare. Im package pathcompareplugins befinden sich die
Pahtcompare Plug-ins. Diese werden, wie im vorigen Abschnitten detailiert beschrieben, zur
Laufzeit dynamisch durch die Hauptanwendung eingebunden.

Ist ein ROS master gestartet, kann Pathcompare durch den ROS Befehl

\begin{lstlisting}[caption=starting pathcompare, label=lst:rosrun]
rosrun pathcompare pathcompare
\end{lstlisting}

ausgeführt werden.
Dieser Befehl sucht dabei automatisch die ausführbare Dabei in der pathcompare
Ordnerstruktur. Alternativ kann diese aber auch manuell ausgeführt werden.
Dies setzt natürlich voraus, dass Pathcompare vorher erfolgreich kompiliert
wurde. Für eine genaue Installationsanleitung siehe dazu den Anhang Installation
oder die in den packages enthaltenen README.txt Dateien.
%TODO readme Dateien


\section{Ausführung des Masters}

Test haben gezeigt, dass als Ausführungsort für den ROS master sich dieselbe
Maschine anbietet, auf der auch Pathcompare läuft. Der Grund dafür ist, dass
Pathcompare häufige Anfragen an dem master, zum auffrischen der Topic
Übersicht, stellt. Läuft der master nicht lokal, sollte zumindest für eine
Netzwerkanbindung mit möglichst niedriger Latenz gesorgt werden. Es kann
ansonsten dazu kommen, dass Teile der GUI blockiert sind. Es wurde versucht
diesem Problem durch caching entgegenzusteuern, indem stets eine Liste der
aktuellen topics lokal im Programm vorgehalten wird. Die Liste soll dabei in einem
separaten Thread regelmäßig aktualisiert werden werden. Allerdings reichte die
Entwicklungszeit nicht aus um diese Änderung in die derzeitige Version
fehlerfrei einzuarbeiten und ist deshalb noch nicht integriert. Siehe dazu auch
Kapitel Fazit.


\section{Konfiguration der Sensorknoten und Roboters}

Wie im Kapitel Implementierung beschrieben wertet das Plug-in Main Compare,
nav\_msgs::path messages aus. Diese werden von den Sensorknoten und dem Roboter
bereitgestellt. Wann diese messages nach dem Start von Pathcompare gesendet
werden kann beliebig bestimmt werden. Optimal ist es jedoch, wenn der Roboter
und die Sensorknoten bereits während der Fahrt in regelmäßigen Abständen
aktualisierte Pfadnachrichten versenden.  Dadurch erhält der Tester schon
während der Testdurchführung von Main Compare ein Feedback welche Tendenzen
sich durch die gemessenen Werte abzeichnen.

\section{Tests}

\subsection{Topic Übersich}
Die Topic Übersicht innerhalb des Rahmens ließ sich unkompliziert Testen, nämlich durch
starten verschiedener ROS nodes. Diese nodes stellen verschiedene topics zur
Verfügung, welche in der Topic Übersicht eingeblendet werden sollen. Dies
funktioniert zuverlässig und es konnten keine Probleme mit der Visualisierung
festgestellt weden.

\subsection{Abbonieren von topics durch Plug-ins}
Das Abbonieren wird, wie im Abschnitt Implementierung genauer beschrieben, durch die
Klasse ROSManager vorgenommen. Für die bisher entwickelten beiden Plug-ins Main
Compare und Camera View traten während durchgeführter Defekttests keine
Laufzeitfehler auf.

\subsection{Main Compare}
Eine Schwierigkeit beim Testen von Main Compare bestand zunächst darin,
geeignete Eingabedaten in Form von path messages zu erhalten. Es konnten jedoch
Pfaddaten, die durch Testfahren des Roboters entstanden sind,
genutzt werden. Dadurch konnten während durchgeführtet Defekttest einige Fehler
identifiziert und beseitigt werden. Allerdings stehen Tests mit sich
kontinuierlich ändernden Pfaddaten z.B. von den Sensorknoten noch aus.

