\chapter{Anwendung}
\label{sec:andwendung}
In diesem Abschnitt soll ein typischer Anwendungsfall für Pathcompare und das
Main Compare Plug-in beschrieben werden. 

\section{Ausführung Pathcompare}

Pathcompare ist in die ROS packages pathcompare und pathcompareplugins
aufgeteilt. Die ausführbare Datei der Pathcompare Hauptanwendung befindet sich
im package pathcompare. Im package pathcompareplugins befinden sich die
Pathcompare Plug-ins. Diese werden, wie im vorigen Abschnitten detailliert
beschrieben, zur Laufzeit dynamisch durch die Hauptanwendung eingebunden.

Ist ein ROS master gestartet, kann Pathcompare durch folgenden den ROS Befehl
ausgeführt werden.

\begin{lstlisting}[caption=starting pathcompare, label=lst:rosrun]
rosrun pathcompare pathcompare
\end{lstlisting}

Dieser Befehl sucht dabei automatisch die ausführbare Dabei in der pathcompare
Ordnerstruktur. Alternativ kann diese aber auch manuell ausgeführt werden.
Dies setzt natürlich voraus, dass Pathcompare vorher erfolgreich kompiliert
wurde. 

Die packages pathcompare und pathcompareplugins können über folgende Quellen bezogen
werden:

pathcompare:
\url{git://github.com/grimrpr/pathcompare.git}

pathcompareplugins:
\url{git://github.com/grimrpr/pathcompareplugins.git}


Für eine genaue Installationsanleitung siehe dazu 
die in den packages enthaltenen README.txt Dateien.
%TODO readme Dateien



\section{Ausführung des Masters}

Tests haben gezeigt, dass als Ausführungsort für den ROS master sich dieselbe
Maschine anbietet, auf der auch Pathcompare läuft. Der Grund dafür ist, dass
Pathcompare häufige Anfragen an dem master, zum Auffrischen der Topic
Übersicht stellt. Läuft der master nicht lokal, sollte zumindest für eine
Netzwerkanbindung mit möglichst niedriger Latenz gesorgt werden. Es kann
ansonsten dazu kommen, dass Teile der GUI blockiert werden. Es wurde versucht
diesem Problem durch caching entgegenzusteuern, indem stets eine Liste der
aktuellen topics lokal im Programm vorgehalten wird. Die Liste soll dabei in einem
separaten Thread regelmäßig aktualisiert werden. Allerdings reichte die
Entwicklungszeit nicht aus, um diese Änderung in die derzeitige Version
fehlerfrei einzuarbeiten und ist deshalb noch nicht integriert. 


\section{Konfiguration der Sensorknoten und des Roboters}

Wie im Kapitel Implementierung beschrieben, wertet das Plug-in Main Compare
nav\_msgs::path messages aus. Diese werden von den Sensorknoten und dem Roboter
bereitgestellt. Wann diese messages nach dem Start von Pathcompare gesendet
werden, kann beliebig bestimmt werden. Optimal ist es, wenn der Roboter
und die Sensorknoten bereits während der Fahrt in regelmäßigen Abständen
aktualisierte Path messages versenden.  Dadurch erhält der Tester schon
während der Testdurchführung von Main Compare ein Feedback, welche Tendenzen
sich durch die gemessenen Werte abzeichnen.

\section{Tests}

\subsection{Topic Übersicht}
Die Topic Übersicht innerhalb des Rahmens ließ sich unkompliziert Testen, nämlich durch
starten verschiedener ROS nodes. Diese nodes stellen verschiedene topics zur
Verfügung, welche in der Topic Übersicht eingeblendet werden sollen.

\subsection{Abonnieren von topics durch Plug-ins}
Das Abonnieren wird, wie im Abschnitt Implementierung genauer beschrieben, durch die
Klasse ROSManager vorgenommen. Für die zwei bisher entwickelten Plug-ins Main
Compare und Camera View traten während durchgeführter Defekttests keine
Laufzeitfehler auf.

\subsection{Main Compare}
Eine Schwierigkeit beim Testen von Main Compare bestand zunächst darin,
geeignete Eingabedaten in Form von path messages zu erhalten. 
Dieses Problem wurde durch die Nutzung von gespeicherten Pfaddaten, die bei Testfahrten des
Roboters entstanden sind, gelöst. 
Dadurch konnten während durchgeführter Defekttests einige Fehler
identifiziert und beseitigt werden. Allerdings müssen noch Tests während des
laufenden Betriebs des Referenzsystems und der Sensorknoten durchgeführt werden.

